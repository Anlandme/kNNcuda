\documentclass{beamer}

\usepackage[utf8]{inputenc}
\usepackage[T1]{fontenc}
\usepackage[english]{babel}
\usepackage{lmodern}
\usepackage{algorithm}
\usepackage{algorithmic}

\usetheme{UOS}

\graphicspath{{images/}}

\renewcommand{\baselinestretch}{1.25}

\begin{document}

\title{k Nearest Neighbour Search with CUDA}
\author[Matthias Greshake, Alexander Mock]{Matthias Greshake, Alexander Mock\\ {\scriptsize mgreshake@uos.de, amock@uos.de}}
\institute{Institut für Informatik\\ AG Wissensbasierte Systeme}
\date{26. January 2016}

\begin{frame}[plain]
	\titlepage
\end{frame} 

\begin{frame}{Outline}
	\tableofcontents
\end{frame} 

\section{Motivation}

\begin{frame}
	\frametitle{Motivation}
\end{frame}

\section{Problem Statement}

\begin{frame}
	\frametitle{Problem Statement}
\end{frame}

\section{Concepts \& Methods}

\begin{frame}
	\frametitle{Parallel kNN Search}
	\begin{columns}[T]
		\begin{column}{0.5\textwidth}
			\textbf{Brute Force}
			\begin{itemize}
				\item simply implementable
				\item highly parallelisable
				\item low memory requirements
				\item quadratic runtime
			\end{itemize}
		\end{column}
		\begin{column}{0.5\textwidth}
			\textbf{kd-Tree}
			\begin{itemize}
				\item hard to implement
				\item only search part parallelisable
				\item memory-intensive
				\item linear runtime
			\end{itemize}
		\end{column}
	\end{columns}
\end{frame}

\subsection*{Brute Force}

\begin{frame}
	\frametitle{Matrix Multiplication}
	image of matrix multiplication
\end{frame}

\begin{frame}
	\frametitle{Pseudocode}
	\begin{algorithmic}
		\FORALL{points in data}
			\STATE calculate distances to all other points
			\REPEAT
				\STATE count neighbours in distance of $\varepsilon$
				\STATE adapt $\varepsilon$
			\UNTIL{number of neighbours in distance $\varepsilon = k$}
			\STATE get neighbours in distance of $\varepsilon$
		\ENDFOR
	\end{algorithmic}
\end{frame}

\subsection*{kd-Tree}

\begin{frame}
	\frametitle{kd-Tree}
\end{frame}

\section{Experiment Results}

\begin{frame}
	\frametitle{Results}
\end{frame}

\section{Conclusion}

\begin{frame}
	\frametitle{Conclusion}
\end{frame}

\end{document}
